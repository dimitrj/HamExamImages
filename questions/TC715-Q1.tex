% Author: Dr. Matthias Jung, DL9MJ
% Year: 2020
\documentclass[convert = false, border=5pt]{standalone}
\usepackage{fontspec}
\setmainfont{Roboto}
\usepackage[siunitx, straightvoltages]{circuitikzgit}
\usepackage{tikz}



\begin{document}

\begin{circuitikz}
    %\tikzstyle{help lines}=[blue!50];
    %\draw[style=help lines] (0,0) grid (8,6);
    %
    \ctikzset{tripoles/en amp/input height=0.45}
    \draw (4,3) node[en amp](E){};
    %(E.out) node[right] {$v_{\mathrm{out}}$} 

    \draw (0,0) to [short, o-o] ++(6,0);
    \draw (E.+) to [short] ++(0,-2.5) node [circ](C1){};
    \draw (C1) node[ground] (gnd2) {};
    %
    \draw(E.-) to [R, -o] (0,3.5);
    \draw (1.5,4.0) node {$\mbox{R}_{\mbox{1}}$};
    \draw (-0.3,3.5) node {E};
    %
    \draw (2.8,3.5) node [circ](C2){}
          to [short] (2.8, 5)
          to [R] ++ (2.5,0)
          to [short, -*] ++ (0,-2);
    \draw (4,5.5) node {$\mbox{R}_{\mbox{2}}$};
    %
    \draw (E.out) to [short, -o] (6,3);
    \draw (6.3,3) node {A};
    %
    \draw (0,3) to [short] (0,0.5) node [currarrow, rotate=-90]{};
    \draw (6,2.5) to [short] (6,0.5) node [currarrow, rotate=-90]{};
    %
    \draw (-0.4,1.75) node {$\mbox{U}_{\mbox{E}}$};
    \draw ( 6.4, 1.5) node {$\mbox{U}_{\mbox{A}}$};
\end{circuitikz}
\end{document}
