% Author: Dr. Matthias Jung, DL9MJ
% Year: 2020
\documentclass[convert = false, border=5pt]{standalone}
\usepackage{fontspec}
\setmainfont{Roboto}
\usepackage[siunitx, straightvoltages]{circuitikzgit}
\usepackage{tikz}


\usepackage[siunitx, straightvoltages]{circuitikzgit}
\usepackage{amsmath}
\usepackage{unicode-math}
\setmathfont{Fira Math}
\setmathfont[range=up]{Roboto}
\setmathfont[range=it]{Roboto-Italic}
\setmathfont[range=\int]{Fira Math}
\usepackage[euler]{textgreek}
\begin{document}
\begin{circuitikz}[straight voltages]
    %\tikzstyle{help lines}=[blue!50];
    %\draw[style=help lines] (0,0) grid (16,8);

    \draw(0,6) node[ocirc] (U1A) {};
    \draw(0,0) node[] (U1B) {};

    \draw(U1A) to [short, i=$\textrm{I}$] ++(2,0)
               to [european resistor, -*, v>=$\textrm{U}_{\textrm{1}}$,
               l=$\textrm{R}_{\textrm{1}}$] ++(0,-3) coordinate(foo)
               to [european resistor, v>=$\textrm{U}_{\textrm{2}}$,
               l=$\textrm{R}_{\textrm{2}}$, i=$\textrm{I}_{\textrm{2}}$] ++(0,-3)
               to [short, -o] ++(-2,0);

    \draw(foo) to [short] ++(2,0)
               to [european resistor,l=$\textrm{R}_{\textrm{L}}$, i=$\textrm{I}_{\textrm{L}}$] ++(0,-3)
               to [short, -*] ++(-2,0);


    \draw(U1A) to[open, v=$\textrm{U}_{\textrm{B}}$] (U1B);

\end{circuitikz}
\end{document}

