% Author: Dr. Matthias Jung, DL9MJ
% Year: 2020
\documentclass[convert = false, border=5pt]{standalone}
\usepackage{fontspec}
\setmainfont{Roboto}
\usepackage[siunitx, straightvoltages]{circuitikzgit}
\usepackage{tikz}



\begin{document}

\begin{circuitikz}
    %\tikzstyle{help lines}=[blue!50];
    %\draw[style=help lines] (0,-2) grid (18,8);
    \draw(0,3) node[antenna]{}
    to [bandpass, box, >, l_={3...30\,MHz}] ++ (3,0);
    \draw(3.5,3) node[mixer, box] (mixx) {} ++(0.5,0)
    to [bandpass, box, >, l_={50\,MHz}] ++ (3,0);
    \draw(mixx.west) node[inputarrow] {} to [short] ++(-1.0,0);
    \draw(mixx.south) node[inputarrow, rotate=90] {} to [short] ++(0,-1);
    \draw(7.5,3) node[mixer, box] (mixy) {} ++(0.5,0)
    to [bandpass, box, >, l_={9\,MHz}] ++ (3,0);
    \draw(mixy.west) node[inputarrow] {} to [short] ++(-1.0,0);
    \draw(mixy.south) node[inputarrow, rotate=90] {} to [short] ++(0,-1);
    \draw(11.5,3) node[mixer, box] (mixz) {} ++(0.5,0)
    to [bandpass, box, >, l_={455\,kHz}] ++ (3,0)
    to [short, -o, l_=3.ZF] ++(0.0,0);
    \draw(mixz.west) node[inputarrow] {} to [short] ++(-1.0,0);
    \draw(mixz.south) node[inputarrow, rotate=90] {} to [short] ++(0,-1);
    %
    \draw(3,1) to [twoport, text=VFO] (4,1);
    \draw(11,1) to [twoport] (12,1);
    \draw(7,1) to [twoport] (8,1);
    \draw(5.5,3.75) node[] {1.ZF};
    \draw(9.5,3.75) node[] {2.ZF};
    \draw(7.3,1.2) node[] {G};
    \draw(11.3,1.2) node[] {G};

    \def\x{0.08}

    \draw[] (7.55,1.3) sin ++(\x, \x)
                       cos ++(\x,-\x)
                       sin ++(\x,-\x)
                       cos ++(\x, \x);

    \draw[] (7.55,1.15)sin ++(\x, \x)
                       cos ++(\x,-\x)
                       sin ++(\x,-\x)
                       cos ++(\x, \x);

    \draw[] (7.55,1.0) sin ++(\x, \x)
                       cos ++(\x,-\x)
                       sin ++(\x,-\x)
                       cos ++(\x, \x);

    \draw[] (11.55,1.3) sin ++(\x, \x)
                        cos ++(\x,-\x)
                        sin ++(\x,-\x)
                        cos ++(\x, \x);

    \draw[] (11.55,1.15)sin ++(\x, \x)
                        cos ++(\x,-\x)
                        sin ++(\x,-\x)
                        cos ++(\x, \x);

    \draw[] (11.55,1.0) sin ++(\x, \x)
                        cos ++(\x,-\x)
                        sin ++(\x,-\x)
                        cos ++(\x, \x);

    \draw(7.2,0.60) -- ++(0.6,0);
    \draw(7.2,0.65) rectangle ++ (0.6,0.1);
    \draw(7.2,0.80) -- ++(0.6,0);

    \draw(11.2,0.60) -- ++(0.6,0);
    \draw(11.2,0.65) rectangle ++ (0.6,0.1);
    \draw(11.2,0.80) -- ++(0.6,0);
    %
\end{circuitikz}
\end{document}
