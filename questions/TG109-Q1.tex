% Author: Dr. Matthias Jung, DL9MJ
% Year: 2020
\documentclass[convert = false, border=5pt]{standalone}
\usepackage{fontspec}
\setmainfont{Roboto}
\usepackage[siunitx, straightvoltages]{circuitikzgit}
\usepackage{tikz}



\begin{document}

\begin{circuitikz}
    %\tikzstyle{help lines}=[blue!50];
    %\draw[style=help lines] (0,-2) grid (10,4);
    %
    \draw (-0.2,3) circle (0.2);
    \draw[thick] (-0.4,3.2) -- (-0.4,2.8);
    %
    \draw (0,3) -- (1,3);
    \draw (1,3) to [amp,box] ++(3,0) 
          node[circ]{} -- ++(2,0)
          to [pR] ++(0,-3);
    \draw (6.555,1.5) |- ++(0.455,1.5);
    %
    \draw (1,3) circle (0.15);
    \draw (0.7,2.85) -- (1.3,2.85);
    \draw (1,2.85) |- (9.5,0) to [short, -o] ++(0,0);
    \draw (2.5,2.5) -- (2.5,0) node[circ]{};
    %
    \draw (7,3) to [lowpass, label={3\,kHz}] ++(2,0) to [short, -o] ++(0.5,0);
    \draw (8,2.5) -- (8,0) node[circ]{};
    %
    \draw (4,3) to [stroke diode] ++(0,-3) node[circ]{};
    \draw (5,0) node[circ]{} to [stroke diode] ++(0,3) node[circ]{};
    %
    \draw (9.5,1.8) node[] {zum};
    \draw (9.5,1.2) node[] {Oszillator};
    %
    \draw (2.5,3.77) node[]{0,3\,-\,3,4\,kHz};
\end{circuitikz}
\end{document}
