% Author: Dr. Matthias Jung, DL9MJ
% Year: 2021
\documentclass[convert = false, border=5pt]{standalone}
\usepackage{fontspec}
\setmainfont{Roboto}
\usepackage[siunitx, straightvoltages]{circuitikzgit}
\usepackage{tikz}


\usepackage[siunitx, straightvoltages]{circuitikzgit}
\usepackage{amsmath}
\usepackage{unicode-math}
\setmathfont{Fira Math}
\setmathfont[range=up]{Roboto}
\setmathfont[range=it]{Roboto-Italic}
\setmathfont[range=\int]{Fira Math}
\usepackage[euler]{textgreek}
\usetikzlibrary{arrows, arrows.meta}

\begin{document}
\begin{circuitikz}
    \draw [dashed] (0,0) -- ++(0.55,0) coordinate(a);
    \draw (a) to [short,o-o] ++(0.5,0)
              to [short,-*] ++(1,0) coordinate(b);
    \ctikzset{inductors/coils=2, inductors/scale=0.5, capacitors/scale=0.5}
    \draw(b)  to [short] ++(0,0.35);
    \draw(b)  to [open] ++(0.5,0.35)
              arc(0:180:0.125)
              arc(0:180:0.125);
    \draw(b)  to [open] ++(0.5,0.35)
              to [short] ++(0,-0.35);
    \draw(b)  to [short] ++(0,-0.35)
              to [C] ++(0.5,0)
              to [short,-*] ++(0,+0.35)
              to [short,-*] ++(2,0) coordinate(x1)
              to [open] ++(0.5,0) coordinate(x2)
              to [short,*-*] ++(2,0) coordinate(c);
    \draw(c)  to [short] ++(0,0.35);
    \draw(c)  to [open] ++(0.5,0.35)
              arc(0:180:0.125)
              arc(0:180:0.125);
    \draw(c)  to [open] ++(0.5,0.35)
              to [short] ++(0,-0.35);
    \draw(c)  to [short] ++(0,-0.35)
              to [C] ++(0.5,0)
              to [short,-*] ++(0,+0.35)
              to [short,-o] ++(1,0)
              to [short,-] ++(0.5,0) coordinate(d);
    \draw [dashed] (d) -- ++(0.55,0);
    \draw (d) to [short,-o] ++(0.0,0);

    \draw (x1) to [open] ++(0,-0.35) node[]{X};
    \draw (x2) to [open] ++(0,-0.35) node[]{X};

    % Lange Striche:
    \draw (1.05,0.15) -- ++ (0,1);
    \draw (4.55,0.15) -- ++ (0,1);
    \draw (5.05,0.15) -- ++ (0,1);
    \draw (8.55,0.15) -- ++ (0,1);

    % Kurze Striche:
    \draw (2.05,0.55) -- ++ (0,0.6);
    \draw (2.55,0.55) -- ++ (0,0.6);
    \draw (7.05,0.55) -- ++ (0,0.6);
    \draw (7.55,0.55) -- ++ (0,0.6);

    % Pfeile:
    \draw[>=triangle 60, <->] (1.05,0.95) -- ++(1,0);
    \draw[>=triangle 60, <->] (2.55,0.95) -- ++(2,0);
    \draw[>=triangle 60, <->] (5.05,0.95) -- ++(2,0);
    \draw[>=triangle 60, <->] (7.55,0.95) -- ++(1,0);

    % Text:
    \draw (1.55,1.35) node[]{6{,}71\,m};
    \draw (3.55,1.35) node[]{10{,}07\,m};
    \draw (6.05,1.35) node[]{10{,}07\,m};
    \draw (8.05,1.35) node[]{6{,}71\,m};
    \draw (2.05,-0.95) node[]{$\mathrm{f}_\mathrm{res}=7{,}05\,\mathrm{MHz}$};
    

\end{circuitikz}
\end{document}

