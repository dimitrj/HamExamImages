% Author: Dr. Matthias Jung, DL9MJ
% Year: 2020
\documentclass[convert = false, border=5pt]{standalone}
\usepackage{fontspec}
\setmainfont{Roboto}
\usepackage[siunitx, straightvoltages]{circuitikzgit}
\usepackage{tikz}



\begin{document}

\begin{circuitikz}
    %\tikzstyle{help lines}=[blue!50];
    %\draw[style=help lines] (0,0) grid (8,6);
    %
    \ctikzset{tripoles/en amp/input height=0.45}
    \draw (4,3) node [en amp](E){};
    \draw (0,0) to   [short, o-o] ++(6,0);
    \draw (E.+) to   [short, -o] (0,2.5);
    %
    %\draw (1.85, 1.5) node {$\mbox{R}_{\mbox{2}}$};
    \draw (1.65, 1.25) node {2,2 kΩ};
    \draw (4.00, 5.6 ) node {22 kΩ};
    %
    \draw (-0.3,2.5) node {E};
    %
    \draw (E.-)
          to [short]     ++(-0.25,0) node [circ](C1){}
          to [short]     ++(0,1.5)
          to [R]         ++(2.75,0)
          to [short, -*] ++(0,-2);
    %
    \draw (C1) to [short] ++(0,-1.0)
               to [R, -*] ++(0,-2.5)
               node [ground] (gnd2) {};
    %
    \draw (E.out) to [short, -o] (6,3);
    \draw (6.3,3) node {A};
    %
    \draw (0,2) to [short] (0,0.5) node [currarrow, rotate=-90]{};
    \draw (6,2.5) to [short] (6,0.5) node [currarrow, rotate=-90]{};
    %
    \draw (-0.4,1.25) node {$\mbox{U}_{\mbox{E}}$};
    \draw ( 6.4, 1.5) node {$\mbox{U}_{\mbox{A}}$};
\end{circuitikz}
\end{document}
