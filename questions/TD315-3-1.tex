% Author: Dr. Matthias Jung, DL9MJ
% Year: 2020
\documentclass[convert = false, border=5pt]{standalone}
\usepackage{fontspec}
\setmainfont{Roboto}
\usepackage[siunitx, straightvoltages]{circuitikzgit}
\usepackage{tikz}


\usepackage{amsmath}
\usepackage{unicode-math}
\setmathfont{Fira Math}
\setmathfont[range=up]{Roboto}
\setmathfont[range=it]{Roboto-Italic}
\setmathfont[range=\int]{Fira Math}
\usepackage[euler]{textgreek}

\begin{document}
\ctikzset{%
}%

\begin{circuitikz}[american]
    %\tikzstyle{help lines}=[blue!50];
    %\draw[style=help lines] (-6,-4) grid (16,4);
    \draw (0,0) node[transformer, yscale=0.95](T){};
    %%
    \draw (T.A1) node [ocirc](I1){};
    \draw (T.A2) node [ocirc](I2){};
    \draw (4,1) node[circ](C1){} to [stroke diode] (T.B1);
    %\draw (T.B1) to [stroke diode] ++(3,0) node[circ](C1){};
    \draw (C1) to [short, -o] ++(1,0);
    \draw (T-L2.south) to [short, *-o] (5,0);
    %
    \draw(C1) to [short] ++(0,-2)
              to [stroke diode] (T.B2);
    %\draw(T.B2) to [stroke diode] ++(3,0) to [short] (C1);
    %
    \draw(5.25,   1) node[]{+};
    \draw(5.25,-0.05) node[]{--};
    %
    \draw (0,0.5) to [short] (0,-0.5);

    \def\x{0.15}
    \draw[] (-1.3,0) sin ++(\x, \x)
                   cos ++(\x,-\x)
                   sin ++(\x,-\x)
                   cos ++(\x, \x);
\end{circuitikz}
\end{document}

