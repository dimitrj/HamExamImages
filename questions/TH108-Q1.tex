% Author: Dr. Matthias Jung, DL9MJ
% Year: 2020
\documentclass[convert = false, border=5pt]{standalone}
\usepackage{fontspec}
\setmainfont{Roboto}
\usepackage[siunitx, straightvoltages]{circuitikzgit}
\usepackage{tikz}


\usepackage[siunitx, straightvoltages]{circuitikzgit}
\begin{document}
\begin{circuitikz}
    %\tikzstyle{help lines}=[blue!50];
    %\draw[style=help lines] (0,0) grid (8,6);
    \draw[ultra thick] (0,2) sin ++(4, 3)
                             cos ++(4,-3);
    \draw[ultra thick] (0,2) sin ++(2, 1.5)
                             cos ++(2,-1.5) 
                             sin ++(2,-1.5)
                             cos ++(2, 1.5);
    \draw[ultra thick] (0,2) sin ++(1, 0.75)
                             cos ++(1,-0.75) 
                             sin ++(1,-0.75)
                             cos ++(1, 0.75)
                             sin ++(1, 0.75)                            
                             cos ++(1,-0.75)
                             sin ++(1,-0.75)
                             cos ++(1, 0.75);

    \draw[ultra thick] (0,2) sin ++(0.5, 0.35)
                             cos ++(0.5,-0.35) 
                             sin ++(0.5,-0.35)
                             cos ++(0.5, 0.35)
                             sin ++(0.5, 0.35)                            
                             cos ++(0.5,-0.35)
                             sin ++(0.5,-0.35)
                             cos ++(0.5, 0.35)
                             sin ++(0.5, 0.35)
                             cos ++(0.5,-0.35) 
                             sin ++(0.5,-0.35)
                             cos ++(0.5, 0.35)
                             sin ++(0.5, 0.35)                            
                             cos ++(0.5,-0.35)
                             sin ++(0.5,-0.35)
                             cos ++(0.5, 0.35);

    \draw[ultra thick] (0,2) to [short] ++( 3.9,   0)
                             to [short] ++(   0,-0.8);
    \draw[ultra thick] (8,2) to [short] ++(-3.9,   0)
                             to [short] ++(   0,-0.8);

    \draw (0,2) to [short] ++ (0,-2);
    \draw (8,2) to [short] ++ (0,-2);
    \draw (0.1, 0) node[inputarrow, rotate=180]{} -- (7.9,0.0) node[inputarrow]{};
    \draw (4,-0.3) node[]{40\,m};
    %
    \draw (6.8,3.8) node[]{D};
    \draw (3.3,3.3) node[]{C};
    \draw (2.5,1.1) node[]{B};
    \draw (1.2,1.5) node[]{A};
\end{circuitikz}
\end{document}
