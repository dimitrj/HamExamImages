% Author: Dr. Matthias Jung, DL9MJ
% Year: 2020
\documentclass[convert = false, border=5pt]{standalone}
\usepackage{fontspec}
\setmainfont{Roboto}
\usepackage[siunitx, straightvoltages]{circuitikzgit}
\usepackage{tikz}



\begin{document}

\begin{circuitikz}
    %\tikzstyle{help lines}=[blue!50];
    %\draw[style=help lines] (0,-2) grid (10,4);
    \draw (-0.2,3) circle (0.2);
    \draw[thick]  (-0.4,3.2) -- (-0.4,2.8);
    \draw (0,3) to [amp,box] ++(3,0);
    \draw (3.5,3) node[mixer, box] (mix) {} ++(1,0);
    \draw (mix.east) to [bandpass, >] ++ (3,0)
                     to node[inputarrow] {} ++(0,0) ;
    \draw (3,1) to [twoport, text=G] ++ (1,0);
    %
    \draw (5.5,-0.1) node[spdt, rotate=-90] (schalter) {}; 
    %
    \draw (5,-2) node[circ]{} -- (5,-1.5) to [piezoelectric,/tikz/circuitikz/bipoles/length=0.7cm, >] ++(0,0.6) to (schalter.out 2);
    \draw (6,-2) -- (6,-1.5) to [piezoelectric,/tikz/circuitikz/bipoles/length=0.7cm, >] ++(0,0.6) to (schalter.out 1);
    \draw (3,1) -- ++(-1,0) -- ++(0,-3) -- ++ (4,0);
    \draw (schalter.in) |- (4,1);
    \draw (3.5,1.5) -- (mix.south) node[inputarrow, rotate=90]{};
    %
    \draw (4.1,2) node[] {$f_{OSZ}$};
    \draw (4.5,3.5) node[] {DSB};
    \draw (6.5,3.5) node[] {SSB};
    \draw (6.5,-0.5) node[] {USB};
    \draw (4.5,-0.5) node[] {LSB};
    \draw (3.5,-1.25) node[] {9,0015\,MHz};
\end{circuitikz}
\end{document}
