% Author: Dr. Matthias Jung, DL9MJ
% Year: 2020
\documentclass[convert = false, border=5pt]{standalone}
\usepackage{fontspec}
\setmainfont{Roboto}
\usepackage[siunitx, straightvoltages]{circuitikzgit}
\usepackage{tikz}



\begin{document}

\begin{circuitikz}
    %\tikzstyle{help lines}=[blue!50];
    %\draw[style=help lines] (0,0) grid (10,4);
    \draw (0,3) 
        to [short, o-] ++(1.5,0)
        to [bandpass, >, label=Filter] ++(2,0)
        to [amp, box, >, label=HF] ++(2,0);
    \draw (6.5,3) node[mixer, box] (mix) {} ++(1,0)
        to [bandpass, >, label=Filter] ++(2,0)
        to [short, -o] ++(0.5,0);
    \draw (0,1) node[ground] (gnd1) {} -- ++(0.45,0) 
        to [PZ, >, label=38.667 MHz] ++(2.05,0)
        to [allpass, >, label=CO] ++(2,0)
        to [twoportsplit, >, label={Vervielfacher~~~~}, t1=1, t2=3] ++(2,0);
    \draw (mix.south) node[inputarrow, rotate=90] {} to [short] ++(0,-1.5);
    \draw (mix.west) node[inputarrow] {} to [short] ++(-0.5,0);
    \draw (mix.east) to [short] ++(0.5,0);
    \draw (9.5,2.15) node[] {28-30\,MHz};
\end{circuitikz}
\end{document}
